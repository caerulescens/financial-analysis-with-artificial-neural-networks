%! Author = caerulescens
%! Date = 4/7/2017

% Preamble
\documentclass[../main.tex]{subfiles}

% Document
\begin{document}
    Seldom reward is absent from risk, and stock markets are a prime example.
    Stock markets across the world are viewed as profitable and risky at the same time.
    Companies have made a business out of forecasting these markets.
    Quantitative analysis companies use mathematicians, financial analysts, and computer scientists to compete in the stock market.
    The old days of floor trading have progressed towards high-frequency trading with supercomputers housed within the exchange.
    For example, the New York Stock exchange has created regulations for these companies so that there’s competitive equality.
    The computer’s power, length of cable to the exchange, and more has been standardized so that no single company will have an advantage with the exception to algorithms.
    Computers are delegated the buying and selling of stocks in the New York Stock exchange.
    A computer receives information from the market, decides an action in microseconds, and that decision gets sent to the exchange in milliseconds.
    From the computer’s perspective, the difference between microseconds and millisecond is significant.
    The company’s trading algorithms are secretive and protected, but their performance depends on time series analysis and machine learning theory.

    Time series analysis is a popular method for forecasting financial systems, but over past decades, machine learning has become an essential area of research with relevant applications in classification and level estimation; both fit into the field of regression analysis within mathematics.
    Classification refers to the labeling of unseen data as a finite number of categories, while level estimation refers to guessing the numeric value of some process.
    For the stock market, classification refers to forecasting the direction of change, while predicting the price of the market is level estimation.

    More generally, this thesis focuses on level estimation of blue chip stocks using artificial neural networks, a type of machine learning model, to forecast next-day closing values.
    Two different optimization methods are investigated to use with artificial neural networks.
    Back-propagation with stochastic gradient descent is the first method, and it is successful at forecasting nonlinear patterns in stock markets.
    The other model is an extreme learning machine, which is a method new to the recent decade.
    Models are trained, validated, and tested for measuring these two methods against each other to discover if extreme learning machines are comparatively successful.

    Each chapter within this paper is purposed with informing the reader about the history, theory, and application of mathematics and computer science techniques for analyzing stocks.
    Chapter two presents the history of time series analysis, definitions, important concepts, and linear and nonlinear forecasting models.
    The structure and components of artificial neural networks are thoroughly explained within chapters three.
    Questions like, ``is the stock market predictable?'' or ``what does academic literature reflect about the success of artificial neural networks in financial markets?'' are answered within chapter four.
    Chapters five and six present the mathematical theory of stochastic gradient descent and extreme learning machines.
    Chapter seven explains the different methods which were used to receive the tabled results of chapter eight.
    Lastly, chapter nine interprets the results from chapter eight, chapter ten provides the thesis’ conclusion, and the appendix contains all graphs and source code.
\end{document}
